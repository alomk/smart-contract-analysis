\section{Foundations}
Blockchain technology's greatest strength is allowing the decentralization of historically centralized services.
Bitcoin~\cite{bitcoin} has a market cap of almost \$200 billion and sees a daily trading volume of about \$100 billion. 
Besides Bitcoin, there are countless other "alt-coins" that each have their own platform and offer a type of service.
Ethereum~\cite{ethereum} is one platform that allows digital assets be directly controlled by a piece of code called a smart contract.
Smart contracts essentially allow systems that automatically move digital assets according to pre-defined rules.
Decentralized autonomous organizations (DAOs) are long-term smart contracts that are controlled by shareholders.
The DAO~\cite{dao}, a crowdfunded venture capital fund that existed as a smart contract, had over \$50 million stolen due to an exploited bug in the smart contract language Solidity.
In fact, this incident is the reason why Ethereum was hard-forked and why we see another alt coin "Ethereum Classic".
Other high profile smart contracts have also been targeted and lost their funds to these types of attacks.
As the usage of smart contracts becomes more prevalent, it is critical to address bugs in smart contract implementations and in Solidity itself.

Smart contract vulnerabilities can be classified~\cite{vulnerabilities} as blockchain vulnerabilities, Solidity vulnerabilities, and software security vulnerabilities.
The Transaction Ordering Dependency problem is one blockchain vulnerability which involves a new block on the chain containing multiple transactions invoking the same contract.
There is no certainty in the state of the contract when either individual transaction invokes the contract because the order is not known until after the block is mined.
The Timestamp dependency problem says that a smart contract using timestamps of blocks can potentially be manipulated by malicious miners who modify their local system's time.

Solidity compiler vulnerabilities are bugs within the smart contract's high level language and how it generates the Ethereum Virtual Machine (EVM) bytecode.
There have not been many CVE's published on this which makes the EVM an excellent candidate for fuzzing.

Software security vulnerabilities are those due to erroneus smart contract code and improper Solidity practice.
The DAO attack was victim to a re-entrancing attack where the contract uses an unsafe function call.value() that allows the attacker to take advantage of a callback to recursively withdraw funds from a victim contract.
Other usual security vulnerabilities such as buffer overruns and integer over/underflows also fall under software bugs.

The parity multisig attack was one incident where \$50 million of Ethereum was frozen due to an attacker calling an unprotected function and gaining ownership of a public smart contract library that was imported by other contracts.
The attacker invoked the kill function that removed the contract from the blockchain.
This could have been mitigated by using a private modifier on the function that allowed the attacker to escalate privileges.

Ethereum smart contracts rely on "gas", which is additional transaction fee paid to the miner for borrowing their computation ability on the blockchain.
If this gas runs out before an operation is over, the callee contract throws an exception and these must be properly checked by the caller contract.
This applys to all other exceptions that can be raised such as when exceeding stack capacity, or any other unknown system error occurs.
These also fall under software security vulnerabilities that can be mitigated by safe code practice.

