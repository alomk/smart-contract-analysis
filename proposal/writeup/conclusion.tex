\section{Conclusion}
Although we could not exploit any vulnerable contracts, we lay the groundwork for assessing new contracts through a variety of open source tools.
We would have initially fallen for honeypot traps if we had been careless and used Manticore to find vulnerabilities in verified source code.
Pakala worked to find trivial bugs and vulnerabilities in bytecode by maximizing coverage and fuzzing inputs, but applying its results usefully was difficult due to a lack of EVM debugging capability.
Echidna is a more precise fuzzing tool than Pakala that uses information from source code to find most of the common vulnerabilities discussed, but both tools lack the ability to find re-entrancy bugs.
Slither can find re-entrancy bugs, but it has a high false positive rate and does not give concrete inputs to trigger such bugs.

An important next step in the black box analysis of EVM bytecode is to develop a flexible debugger that can step through instructions and insert breakpoints where local memory and blockchain storage can be inspected.
This would make it easy to universally analyze contracts the same as one would reverse engineer an x86 executable file.

Clearly there are novel techniques consistently being innovated to analyze contracts where some are advantageous over others in certain cases and vice versa.
These tools can be leveraged by developers to test their smart contracts before initializing them on the blockchain, but in practice it is recommended to instead use a method called "formal verification" as in the tool Securify~\cite{securify}.
We could not analyze with Securify because it is a proprietary software.
Furthermore, formal verification is a tedious and expensive process as it requires manually writing specifications for the contracts to be tested.
The alternative to formal verification for writing secure smart contract code is to use open source~\cite{openzeppelin} libraries of contracts thoroughly vetted by the community.
For instance, you may copy from a library that provides safe math functions where integer overflows are prevented.
Regardless, vulnerabilities in open source libraries would compromise all the contracts that inherit from them.









