\section{Initial Approach}

The current plan was to use Manticore~\cite{manticore}, a symbolic execution engine that supports EVM bytecode (and other architectures) to maximize code coverage on smart contracts sampled from Etherscan~\cite{etherscan}.
Manticore can simulate the EVM as well as simulate an entire "Ethereum world" where a contract may call on other contracts.
One advantage is that this analysis can be done without any real Ethereum coin because Manticore can set arbitrary amounts for contracts and wallets you define in the API.

The objective was to identify devastating vulnerabilities we have seen before such as the re-entrancing attack on the DAO that allows an attacker to steal funds.
Manticore would provide inputs that will hit various branches in smart contracts and the next step is to apply a fuzzer to user-controlled inputs in order to generate corner cases.
In other words, we want to cause as many exceptions we can along every branch in the code and log them for further analysis.
The collected smart contracts will also be analyzed for semantic bugs after reading up on the latest version of Solidity and recommended practice.

One difficulty in the approach may be dealing with all the versions of Solidity being used in the wild.
Solidity is constantly being patched which causes smart contract writers to be slow in keeping up.
Manticore fully supports Solidity versions pre-0.5.0 so we will have to scrape contracts using Solidity 0.4 and older.


